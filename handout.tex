\documentclass[12pt]{scrartcl}

\setlength{\parindent}{0pt}
\setlength{\parskip}{.25cm}

\usepackage{graphicx}

\usepackage{xcolor}

\definecolor{darkred}{rgb}{0.5,0,0}
\definecolor{darkgreen}{rgb}{0,0.5,0}
\usepackage{hyperref}
\hypersetup{
  letterpaper,
  colorlinks,
  linkcolor=red,
  citecolor=darkgreen,
  menucolor=darkred,
  urlcolor=blue,
  pdfpagemode=none,
  pdftitle={CSCE 156 Lab Handout},
  pdfauthor={Christopher M. Bourke},
  pdfsubject={},
  pdfkeywords={}
}

\definecolor{MyDarkBlue}{rgb}{0,0.08,0.45}
\definecolor{MyDarkRed}{rgb}{0.45,0.08,0}
\definecolor{MyDarkGreen}{rgb}{0.08,0.45,0.08}

\definecolor{mintedBackground}{rgb}{0.95,0.95,0.95}
\definecolor{mintedInlineBackground}{rgb}{.90,.90,1}

%\usepackage{newfloat}
\usepackage[newfloat=true]{minted}
\setminted{mathescape,
               linenos,
               autogobble,
               frame=none,
               framesep=2mm,
               framerule=0.4pt,
               %label=foo,
               xleftmargin=2em,
               xrightmargin=0em,
               startinline=true,  %PHP only, allow it to omit the PHP Tags *** with this option, variables using dollar sign in comments are treated as latex math
               numbersep=10pt, %gap between line numbers and start of line
               style=default, %syntax highlighting style, default is "default"
               			    %gallery: http://help.farbox.com/pygments.html
			    	    %list available: pygmentize -L styles
               bgcolor=mintedBackground} %prevents breaking across pages
               
\setmintedinline{bgcolor={mintedBackground}}
\setminted[text]{bgcolor={mintedBackground},linenos=false,autogobble,xleftmargin=1em}
%\setminted[php]{bgcolor=mintedBackgroundPHP} %startinline=True}
\SetupFloatingEnvironment{listing}{name=Code Sample}
\SetupFloatingEnvironment{listing}{listname=List of Code Samples}

\title{CSCE 156 -- Computer Science II}
\subtitle{Lab 3.0 - Strings \& File I/O}
\author{~}
\date{~}

\begin{document}

\maketitle

\section*{Prior to Lab}

Review this laboratory handout prior to lab.

For Java:
\begin{enumerate}
  \item Read Java String tutorial: \\
	\url{http://download.oracle.com/javase/tutorial/java/data/strings.html}
  \item Read Java Arrays tutorial: \\
	\url{http://download.oracle.com/javase/tutorial/java/nutsandbolts/arrays.html}
  \item Read Java File I/O tutorial: \\
  	\url{http://download.oracle.com/javase/tutorial/essential/io/file.html}
  \item Read the Java tutorial on the \mintinline{java}{System.printf()} 
	method: \\
	\url{https://docs.oracle.com/javase/tutorial/java/data/numberformat.html}
\end{enumerate}

For PHP:
\begin{enumerate}
  \item Read PHP Arrays tutorial: \\
    \url{http://www.php.net/manual/en/language.types.array.php}
  \item Read PHP tutorial on string functions: \\
	\url{http://www.php.net/manual/en/ref.strings.php}
  \item Read PHP tutorial on file streams: \\
	\url{http://www.php.net/manual/en/book.filesystem.php}
  \item Read the PHP Manual on the \mintinline{php}{printf()} 
    (print formatted) and \mintinline{php}{sprintf()} functions:\\
	\url{http://php.net/manual/en/function.printf.php}\\
	\url{http://www.php.net/manual/en/function.sprintf.php}
\end{enumerate}

\section*{Lab Objectives \& Topics}
Following the lab, you should be able to:
\begin{itemize}
  \item Use Strings and Arrays in PHP/Java
  \item Use basic File I/O in PHP/Java
\end{itemize}

\section*{Peer Programming Pair-Up}

To encourage collaboration and a team environment, labs will be
structured in a \emph{pair programming} setup.  At the start of
each lab, you will be randomly paired up with another student 
(conflicts such as absences will be dealt with by the lab instructor).
One of you will be designated the \emph{driver} and the other
the \emph{navigator}.  

The navigator will be responsible for reading the instructions and
telling the driver what to do next.  The driver will be in charge of the
keyboard and workstation.  Both driver and navigator are responsible
for suggesting fixes and solutions together.  Neither the navigator
nor the driver is ``in charge.''  Beyond your immediate pairing, you
are encouraged to help and interact and with other pairs in the lab.

Each week you should alternate: if you were a driver last week, 
be a navigator next, etc.  Resolve any issues (you were both drivers
last week) within your pair.  Ask the lab instructor to resolve issues
only when you cannot come to a consensus.  

Because of the peer programming setup of labs, it is absolutely 
essential that you complete any pre-lab activities and familiarize
yourself with the handouts prior to coming to lab.  Failure to do
so will negatively impact your ability to collaborate and work with 
others which may mean that you will not be able to complete the
lab.  

\section*{Getting Started}

Clone the project code for this lab from GitHub in Eclipse using the
URL, \url{https://github.com/cbourke/CSCE156-Lab03}.
Refer to Lab 1.0 for instructions on how to clone a project from GitHub.

For those with prior Java experience, do the PHP section.  For those
without prior Java experience, do the Java section.

\part*{PHP}

\section*{Strings \& File IO}

You will familiarize yourself with strings and file input/output 
by completing two PHP scripts.  

The first program involves processing a DNA nucleotide sequence (a 
string consisting of the characters A, G, C, and T standing for the 
nucleobases adenine, guanine, cytosine, and thymine).  A common 
operation on DNA is searching for and counting the number of instances 
of a particular subsequence.  For example, in the following DNA 
sequence, 
$$TAGAAAAGGGAAAGATAGT$$
the subsequence $TAG$ appears twice.  Your activity will involve 
processing a file containing a nucleotide sequence of the H1N1 flu 
virus and counting the number of instances of various subsequences.

The second program involves processing a file containing formatted 
data.  Specifically, you will process a file containing the win/loss 
records of National League baseball Teams from the 2011 season.  The 
file is formatted as follows: each line contains the win/loss record 
of a single team (16 teams total).  Each line contains the team name, 
the number of wins, and the number of losses.  Your program will read 
in the file, process the data and sort the teams in the order of their 
win percentage (wins divided by total games) and output the sorted and 
reformatted team list into a new file.

\subsection*{Formatted Output}

Most programming languages support or implement the standard 
functionality of the \mintinline{text}{printf()} standard of formatted 
output originally provided in the C programming language.

The \mintinline{text}{printf()} function is a variable argument 
function that takes a string as its first argument that specifies 
a format in which to print the subsequent arguments.  Various flags 
can be used to print variable values in a specific format or as a 
specific type.  Some of the major flags supported:
\begin{itemize}
  \item \mintinline{text}{%Ns} -- print the argument as a string, 
  	right-justified with at least $N$ characters.  If the string 
	value is less than $N$, it will be padded out with spaces.  
	Variations on this flag can be used to change the padding 
	character and to left-justify (negative $N$) instead.
  \item \mintinline{text}{%Nd} -- print the argument as an integer 
	with at least $N$ spaces.
  \item \mintinline{text}{%N.Mf} -- print the argument as a floating 
	point number with at least $N$ characters (including the decimal) 
	and at most $M$ decimals of precision.
\end{itemize}
	
Consider the following example code snippet.

\begin{minted}{php}
$a = "hello";
$b = 42;
$c = 3.1418;
printf("%10s, %5d\t%5.2f\n", $a, $b, $c);
\end{minted}

This would result in the following (note the spacing denoted
with a \textvisiblespace\ symbol):

\begin{minted}[showspaces]{text}
>     hello,     42	 3.14
\end{minted}     

\section*{Activities}

\subsection*{Substring Searching}

\begin{enumerate}
  \item Open the \mintinline{text}{PHP/dna.php} and the 
  	\mintinline{text}{data/H1N1nucleotide.txt} files
  \item Modify the code to read in a subsequence from the command 
  	line (and to echo an error and exit if it is not provided).
  \item The code to read in and process the nucleotide sequence is 
	already provided.  Study its usage: it reads in the file line by 
	line, trims each line (removes leading and trailing whitespace) and 
	concatenates it into one large string.
  \item At the end of the script, write your own code to count the
	number of occurrences of the subsequences.  Strings in PHP can 
	be processed like character arrays: \mintinline{php}{$subsequence[0]} 
	gives you the first character of \mintinline{php}{$subsequence}. 
	Individual characters can be compared using the \mintinline{php}{==}
	comparison operator.
  \item Hand in your source file using the webhandin and grade yourself 
	with the webgrader.
\end{enumerate}
	
\subsection*{Baseball Records}

\begin{enumerate}
  \item Open the \mintinline{text}{PHP/Team.php} and 
  	\mintinline{text}{PHP/baseball.php} files.  The \mintinline{php}{Team} 
	class has already been implemented and included in the 
	\mintinline{text}{baseball.php} script.
  \item Much of the code has been provided for you, including code to 
	print the teams for debugging purposes and to sort the teams according 
	to win percentage.
  \item Write code to read in the \mintinline{text}{data/mlb_nl_2011.txt} 
	data file at the appropriate point in this script.  Each line in this file contains a
	team name, number of wins, and number of losses. You may find the following functions
	useful:
  \begin{itemize}
	 \item \mintinline{php}{preg_split($pattern, $str)} (\url{http://php.net/manual/en/function.preg-split.php})
	 \item \mintinline{php}{explode($delimiter, $str)} (\url{http://www.php.net/manual/en/function.explode.php})
  \end{itemize}
  	Both functions tokenize the given string according to a regular 
	expression or a delimiter respectively and return an array of 
	string tokens.  
  \item The \mintinline{php}{Team} class has one constructor that 
	takes the team name, wins, and losses; example:
	
	\mintinline{php}{$team = new Team($name, $wins, $loss);}
	
	You can access the methods of the \mintinline{php}{Team} class using
	the syntax: 
	
	\begin{minted}{php}
	$name = $team->getName();
	$winPerc = $team->getWinPercentage();
	\end{minted}
	
  \item Write code to output the sorted array of teams to a file 
  	\mintinline{text}{data/mlb_nl_2011_results.txt} according 
	to the following format:
	\begin{itemize}
	  \item Each team's information should appear on a separate line.
	  \item Each line should contain the team's name and its win percentage.
      \item The team name should be right-justified in a column of length 
	  	10, the win percentage should be a number 0--100 with two decimals 
		of precision.
	\end{itemize}
	Your output file should look like the following:

\begin{minted}{text}
 Cardinals 55.56
    Braves 54.94
    Giants 53.09
   Dodgers 50.93
 Nationals 49.69
\end{minted}

  \item Answer the questions in your worksheet and demonstrate your 
  	working programs to a lab instructor.
\end{enumerate}

\subsection*{Advanced Activity (Optional)}
	
The code to sort the teams according to their win percentage was 
provided for you.  It involved defining a \emph{comparator} function 
that was passed as an argument to a built-in sort function.  Study 
this code and read the documentation for the sorting function.  
Modify the comparison function to sort the list of teams in alphabetic 
order according to the team name.

\newpage
\part*{Java}

\section*{Strings \& File IO}

You will familiarize yourself with strings and file input/output 
by completing two Java programs.  

The first program involves processing a DNA nucleotide sequence (a 
string consisting of the characters A, G, C, and T standing for the 
nucleobases adenine, guanine, cytosine, and thymine).  A common 
operation on DNA is searching for and counting the number of instances 
of a particular subsequence.  For example, in the following DNA 
sequence, 
$$TAGAAAAGGGAAAGATAGT$$
the subsequence $TAG$ appears twice.  Your activity will involve 
processing a file containing a nucleotide sequence of the H1N1 flu 
virus and counting the number of instances of various subsequences.

The second program involves processing a file containing formatted 
data.  Specifically, you will process a file containing the win/loss 
records of National League baseball Teams from the 2011 season.  The 
file is formatted as follows: each line contains the win/loss record 
of a single team (16 teams total).  Each line contains the team name, 
the number of wins, and the number of losses.  Your program will read 
in the file, process the data and sort the teams in the order of their 
win percentage (wins divided by total games) and output the sorted and 
reformatted team list into a new file.

\subsection*{Formatted Output}

Most programming languages support or implement the standard 
functionality of the \mintinline{text}{printf()} standard of formatted 
output originally provided in the C programming language.

The \mintinline{text}{printf()} function is a variable argument 
function that takes a string as its first argument that specifies 
a format in which to print the subsequent arguments.  Various flags 
can be used to print variable values in a specific format or as a 
specific type.  Some of the major flags supported:
\begin{itemize}
  \item \mintinline{text}{%Ns} -- print the argument as a string, 
  	right-justified with at least $N$ characters.  If the string 
	value is less than $N$, it will be padded out with spaces.  
	Variations on this flag can be used to change the padding 
	character and to left-justify (negative $N$) instead.
  \item \mintinline{text}{%Nd} -- print the argument as an integer 
	with at least $N$ spaces.
  \item \mintinline{text}{%N.Mf} -- print the argument as a floating 
	point number with at least $N$ characters (including the decimal) 
	and at most $M$ decimals of precision.
\end{itemize}
	
Consider the following example code snippet.

\begin{minted}{java}
String a = "hello"; 
int b = 42;
double c = 3.1418;
String result = String.format("%10s, %5d\t%5.2f\n", a, b, c);
System.out.println(result);

//alternatively:
System.out.printf("%10s, %5d\t%5.2f\n", a, b, c);
\end{minted}

This would result in the following (note the spacing denoted
with a \textvisiblespace\ symbol):

\begin{minted}[showspaces]{text}
>     hello,     42	 3.14
\end{minted}     

\section*{Activities}

\subsection*{Substring Searching}

\begin{enumerate}
  \item Open the \mintinline{text}{DNA.java} and the 
  	\mintinline{text}{data/H1N1nucleotide.txt} files
  \item Modify the code to read in a subsequence from the command 
  	line (and to echo an error and exit if it is not provided).
  \item The code to read in and process the nucleotide sequence is 
	already provided.  Study its usage: it reads in the file line by 
	line, trims each line (removes leading and trailing whitespace) and 
	concatenates it into one large string.
  \item At the end of the script, write your own code to count the
	number of occurrences of the subsequences.  Individual characters
	in a Java \mintinline{java}{String} can be retrieved using the
	\mintinline{java}{charAt(int)} method.  For example, 
	\mintinline{java}{subsequence.charAt(0)} gives you the first 
	character of the string \mintinline{java}{subsequence}. 
	Individual characters can be compared using the \mintinline{java}{==}
	comparison operator.
  \item Hand in your source file using the webhandin and grade yourself 
	with the webgrader.
\end{enumerate}
	
\subsection*{Baseball Records}

\begin{enumerate}
  \item Open the \mintinline{text}{Team.java} and 
  	\mintinline{text}{Baseball.java} files.  The \mintinline{java}{Team} 
	class has already been implemented for you.
  \item Much of the code has been provided for you, including code to 
	print the teams for debugging purposes and to sort the teams according 
	to win percentage.
  \item Write code to read in the \mintinline{text}{data/mlb_nl_2011.txt} 
	data file at the appropriate point in this script.  Each line in this file contains a
	team name, number of wins, and number of losses.  You may find the 
	following useful:
  \begin{itemize}
	\item The \mintinline{java}{Scanner} class can be used to get
	 individual tokens in a file.  By default the delimiter for this
	 class is any whitespace.  
	\item The \mintinline{java}{Team} class has one constructor 
	 that takes the team name, wins, and losses; example:
	 
	 \mintinline{java}{Team t = new Team(name, wins, loss);}
	 
    \item If you have an instance of \mintinline{java}{Team}, you can 
	 make use of its methods using the following syntax:
	 
	 \begin{minted}{java}
	 String name = t.getName();
	 double winPerc = t.getWinPercentage();
	 \end{minted}
  \end{itemize}
	
  \item Write code to output the sorted array of teams to a file 
  	\mintinline{text}{data/mlb_nl_2011_results.txt} according 
	to the following format:
	\begin{itemize}
	  \item Each team's information should appear on a separate line.
	  \item Each line should contain the team's name and its win percentage.
      \item The team name should be right-justified in a column of length 
	  	10, the win percentage should be a number 0--100 with two decimals 
		of precision.
	\end{itemize}
	Your output file should look like the following:

\begin{minted}{text}
 Cardinals 55.56
    Braves 54.94
    Giants 53.09
   Dodgers 50.93
 Nationals 49.69
\end{minted}

Hint: to output to a file, use the \mintinline{java}{PrintWriter} 
class which supports easy output to files.  A full example:

\begin{minted}{java}
PrintWriter out;
try {
  out = new PrintWriter("filename");
} catch (FileNotFoundException fnfe) {
  throw new RuntimeException(fnfe);
}
out.write("you can output a string directly using this method!");
out.close();
\end{minted}
  \item Answer the questions in your worksheet and demonstrate your 
  	working programs to a lab instructor.
\end{enumerate}

\subsection*{Advanced Activity (Optional)}
	
The code to sort the teams according to their win percentage was 
provided for you.  It involved defining a \mintinline{java}{Comparator}
(as an anonymous class) that was passed as an argument to a built-in
sort method.  Study this code and read the documentation for the sorting 
method.  Modify the \mintinline{java}{Comparator} to sort the list of 
teams in alphabetic order according to the team name.


\end{document}
